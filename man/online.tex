\documentstyle[manual,11pt,tty,psfig]{report}

\newcommand{\predicate}[3]{\pagebreak\item[#1({\it #3})]\hfil\mbox{}\\}
\newcommand{\prefix}[2]{\pagebreak\item[#1 {\it #2}]\hfil\mbox{}\\}
\newcommand{\infix}[3]{\pagebreak\item[{\it #1} #2 {\it #3}]\hfil\mbox{}\\}
\newcommand{\noargpredicate}[1]{\pagebreak\item[#1]\hfil\mbox{}\\}
\newcommand{\function}[4]{\pagebreak\item[{\it #1}\space] {\bf #2(\it #4})\\}
\newcommand{\noargfunction}[2]{\pagebreak\item[{\it #1}\space] {\bf #2()}\\}
\newcommand{\macro}[4]{\pagebreak\item[{\it #1}\space] {\bf #2(\it #4})\\}
\newcommand{\noargmacro}[2]{\pagebreak\item[{\it #1}\space] {\bf #2}\\}
\newcommand{\variable}[2]{\pagebreak\item[{\it #1}\space] {\bf #2}\\}
\newcommand{\tty}[1]{\mbox{\verb@#1@}}
\newcommand{\bug}[1]{\par BUG: #1}
\newcommand{\version}{1.6.0, May 1992}
\itemsep 0pt

%\makeindex
\sloppy
%\psdraft

%\includeonly{builtin}

\begin{document}

\raggedright

\begin{titlepage}

\newlength{\uvawidth}
\settowidth{\uvawidth}{\LARGE University of Amsterdam}

\newcommand{\uvaaddress}{%
\parbox[b]{\uvawidth}{%
    \begin{center}
	\LARGE
	      University of Amsterdam \\[3mm]
	\small
	Dept. of Social Science Informatics (SWI) \\
	    Herengracht 196, 1016 BS~~ Amsterdam \\
		    The Netherlands \\
		Tel. (+31) 20 5252073
    \end{center}}}

\mbox{}\vspace{-1in}
\parbox{\textwidth}{%
    \makebox[\textwidth]{%
	\UvA{1in}%
	\hfill%
	\raisebox{-12pt}{\uvaaddress}
	\hfill%
	\SWI{1in}}
}
\vfil\vfil\vfil
\begin{center}
	{\Huge \bf SWI-Prolog 1.6	\\[3mm]
	 \LARGE Reference Manual}	\\[1.5cm]
	{\large \it Jan Wielemaker}	\\[7mm]
	{\large jan@swi.psy.uva.nl}
\end{center}
\vfil
\begin{quote}
SWI-Prolog is a WAM (Warren Abstract Machine, \cite{Warren:83b}) based
implementation of Prolog.  SWI-Prolog has been designed and implemented
such that it can easily be modified for experiments with logic
programming and the relation between logic programming and other
programming paradigms (such as the object oriented PCE environment,
\cite{P1098:C1.6}).  SWI-Prolog has a rich set of built-in predicates
and reasonable performance, which makes it possible to develop
substantial applications in it.  The current version offers a module
system, garbage collection and an interface to the C language.

This document gives an overview of the features, system limits and
built-in predicates.
\end{quote}
\vfil
\vfil
\begin{quote}
Copyright \copyright\ 1990 Jan Wielemaker
\end{quote}
\end{titlepage}

\pagestyle{empty}

{\parskip 0pt
\tableofcontents
}

\include{intro}
\include{builtin}
\include{module}
\include{foreign}
\include{hack}
\include{summary}

%\bibliographystyle{name}
%\bibliography{abc,esprit,manual}

%\documentstyle[manual,twoside,makeidx]{report}

\input psfig.sty

\newcommand{\predicate}[3]{\item[#1({\it #3})]\hfil\mbox{}\\}
\newcommand{\prefix}[2]{\item[#1 {\it #2}]\hfil\mbox{}\\}
\newcommand{\infix}[3]{\item[{\it #1} #2 {\it #3}]\hfil\mbox{}\\}
\newcommand{\noargpredicate}[1]{\item[#1]\hfil\mbox{}\\}
\newcommand{\function}[4]{\item[{\it #1}] {\bf #2(\it #4})\\}
\newcommand{\noargfunction}[2]{\item[{\it #1}] {\bf #2()}\\}
\newcommand{\macro}[4]{\item[{\it #1}] {\bf #2(\it #4})\\}
\newcommand{\noargmacro}[2]{\item[{\it #1}] {\bf #2}\\}
\newcommand{\variable}[2]{\item[{\it #1}] {\bf #2}\\}
\newcommand{\tty}[1]{\mbox{\verb@#1@}}
\newcommand{\bug}[1]{\footnote{BUG: #1}}
\newcommand{\version}{1.6.0, May 1992}
\newcommand{\pow}[2]{{#1}^{#2}}
\itemsep 0pt

\makeindex
\sloppy
%\psdraft

%\includeonly{foreign}

\begin{document}

\begin{titlepage}

\newlength{\uvawidth}
\settowidth{\uvawidth}{\LARGE University of Amsterdam}

\newcommand{\uvaaddress}{%
\parbox[b]{\uvawidth}{%
    \begin{center}
	\LARGE
	      University of Amsterdam \\[3mm]
	\small
	Dept. of Social Science Informatics (SWI) \\
	    Roeterstraat 15, 1018 WB~~Amsterdam \\
		    The Netherlands \\
		Tel. (+31) 20 5256786
    \end{center}}}

\mbox{}\vspace{-1in}
\parbox{\textwidth}{%
    \makebox[\textwidth]{%
	\UvA{1in}%
	\hfill%
	\raisebox{-12pt}{\uvaaddress}
	\hfill%
	\SWI{1in}}
}
\vfil\vfil\vfil
\begin{center}
	{\Huge \bf SWI-Prolog 1.6	\\[3mm]
	 \LARGE Reference Manual}	\\[1.5cm]
	{\large \it Jan Wielemaker}	\\[7mm]
	{\large jan@swi.psy.uva.nl}
\end{center}
\vfil
\begin{quote}
SWI-Prolog is a Prolog implementation based on a subset of the WAM
(Warren Abstract Machine, \cite{Warren:83b}). SWI-Prolog has been
designed and implemented such that it can easily be modified for
experiments with logic programming and the relation between logic
programming and other programming paradigms (such as the object
oriented PCE environment, \cite{P1098:C1.6}).  SWI-Prolog has a rich
set of built-in predicates and reasonable performance, which makes it
possible to develop substantial applications in it.  The current
version offers a module system, garbage collection and an interface to
the C language.

This document gives an overview of the features, system limits and
built-in predicates.
\end{quote}
\vfil
\vfil
\begin{quote}
Copyright \copyright\ 1991, 1992 Jan Wielemaker
\end{quote}
\end{titlepage}

\pagestyle{esprit}
\newcommand{\bottomleft}{\mbox{}}
\newcommand{\bottomright}{\mbox{}}

{\parskip 0pt
\tableofcontents
}

\include{intro}
\include{builtin}
\include{module}
\include{foreign}
\include{hack}
\include{summary}

\bibliographystyle{name}
\bibliography{abc,esprit,manual}

\printindex

\end{document}


\end{document}
